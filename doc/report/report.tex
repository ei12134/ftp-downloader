\documentclass[a4paper,11pt,titlepage]{article}
\usepackage[utf8]{inputenc}
\usepackage{lmodern} \usepackage[T1]{fontenc}
\usepackage[babel=true]{microtype}
\usepackage[portuguese]{babel}
\usepackage[pdftex]{hyperref}
\usepackage{graphicx}
\usepackage{eurosym}
\usepackage{scrextend}
\usepackage{hyphenat}
\usepackage{url}
\usepackage{hyperref}
\usepackage{listings}
\usepackage{indentfirst}
\usepackage{float}
\usepackage[usenames,dvipsnames,svgnames,table]{xcolor}

\lstdefinestyle{customc}{
  belowcaptionskip=1\baselineskip,
  breaklines=true,
  xleftmargin=\parindent,
  language=C,
  showstringspaces=false,
  tabsize=2,
  basicstyle=\footnotesize\ttfamily,
  keywordstyle=\bfseries\color{blue},
  commentstyle=\itshape\color{gray},
  identifierstyle=\color{black},
  stringstyle=\color{OliveGreen},
}

\lstdefinestyle{customcwithlines}{
  belowcaptionskip=1\baselineskip,
  breaklines=true,
  xleftmargin=\parindent,
  language=C,
  showstringspaces=false,
  numbers=left,
  tabsize=2,
  basicstyle=\footnotesize\ttfamily,
  keywordstyle=\bfseries\color{blue},
  commentstyle=\itshape\color{gray},
  identifierstyle=\color{black},
  stringstyle=\color{OliveGreen},
}

\title{\huge \textbf{Aplicação de download e configuração e estudo uma rede\\[1cm] \Large Relatório\\[0.7cm]
\includegraphics{res/logo.png}\\[0.7cm] \large Redes de Computadores\\[0.25cm] \small $3^o$ ano\\[0.05cm]Mestrado Integrado em Engenharia Informática e
Computação\\[1cm]}\normalsize Turma 4}

\author{Carolina Moreira\\Daniel Fazeres\\José Peixoto \and 201303494\\201502846\\200603103 \and  up201303494@fe.up.pt\\up201502846@fe.up.pt\\ei12134@fe.up.pt}

\begin{document}
\maketitle

\newpage
\tableofcontents
\newpage

\abstract
\iffalse dois parágrafos: um sobre o contexto do trabalho; outro sobre as principais conclusões do relatório \fi

No âmbito da unidade curricular de Redes de Computadores, foi-nos proposto o desenvolvimento de uma aplicação que testasse um protocolo de ligação de dados criado de raiz, transferindo um ficheiro recorrendo à porta de série $RS-232$. O trabalho permitiu praticar conceitos teóricos no desenho de um protocolo de ligação de dados como o sincronismo e delimitação de tramas, controlo de erros, controlo de fluxo recurso a mecanismos de transparência de dados na transmissão assíncrona.

Findo o projeto, notou-se a importância dos mecanismos que asseguram tolerância a falhas fornecidos pela camada de ligação de dados, uma vez que a camada física não é realmente fiável.

\section{Introdução}
\iffalse(indicação dos objectivos do trabalho e do relatório; descrição da lógica do relatório com indicações sobre o tipo de informação que poderá ser encontrada em cada uma secções seguintes) \fi
O objetivo do trabalho realizado nas aulas laboratoriais da disciplina de Redes de Computadores é a implementação de um protocolo de ligação de dados que permita praticar conhecimentos acerca de transmissões de dados entre computadores, programando em baixo nível as características comuns a este tipos de protocolos como a transparência na transmissão de dados de forma assíncrona e organização da informação sob a forma de tramas.

\section{Aplicação de download}
A primeira parte do segundo trabalho laboratorial consistiu no desenvolvimento de uma aplicação de download recorrendo ao protocolo de transferência de ficheiros FTP especificado pelo \texttt{RFC959}. Para este efeito foi também necessário resolver o endereço de IP para um dado URL de acordo com especificação RFC1738. Separaram-se as componentes do parsing do URL e do cliente de download usando o protocolo FTP respectivamente nos ficheiros \texttt{url.c} e \texttt{ftp.c}.

\subsection*{Casos de uso principais}
\iffalse (identificação; sequências de chamada de funções) \fi
O programa implementa uma versão básica de um cliente de FTP com suporte para download de ficheiros de forma anónima ou para um dado par utilizador e password, introduzidos antecipadamente ao caminho de URL do ficheiro. 
Apesar da introdução do nome de utilizador seguido de password serem facultativos, em alguns casos tornam-se obrigatórios na transferência com sucesso de um ficheiro por FTP caso este não esteja disponível de forma pública e requeira autenticação por parte do utilizador.

\begin{figure}[H]
    \center
    \includegraphics[scale=0.6]{res/usage.png}
    \caption{Utilização do programa}
    \label{fig:usage.png}
\end{figure}

\subsection*{Análise de URL}
Após a leitura e compreensão do \texttt{RFC1738}, desenhou-se uma estrutura de dados com a finalidade de armazenar a informação extraída da análise de um link URL recebido pela linha de comandos. Quando fornecidos, são guardados na estrutura referida, o nome de utilizador, a password, o host, ip, path e nome do ficheiro em arrays de caracteres independentes. Para além disso, é também definida a porta 21 como a predefinida na ligação de controlo do protocolo de FTP. À semelhança do que é referido no \texttt{RFC1738}, caracteres capitalizados são interpretados como caracteres minúsculos, admitindo ftp da mesma forma que FTP.

Para se poder resolver o endereço de IP armazenado na estrutura de URL, é feita uma chamada à função \texttt{gethostbyname} que permite a obtenção do endereço de uma máquina a partir do nome e retorna uma estrutura do tipo \texttt{hostent} contendo o endereço na variável \texttt{h\_addr} que é posteriormente convertido para um array de chars com o auxílio da função \texttt{inet\_ntoa}.

\subsection*{Cliente de FTP}
O cliente de FTP liga-se através de um socket TCP ao servidor de FTP identificado pelo endereço de IP acima mencionado e à porta 21 e estabelece uma ligação de controlo de comunicação, comunicando com uma sequência de comandos de transferência FTP que lhe são enviados ao estilo do protocolo Telnet: 

\begin{labeling}{alligator}
\item [\textbf{USER user}] envio do nome de utilizador sob a forma de uma string que identifica o utilizador no servidor remoto.
\item [\textbf{PASS pass}] envio da palavra passe completando a identificação no sistema de identificação e controlo de acesso do servidor.
\item [\textbf{CWD path}] indicação do directório que contém o ficheiro requerido para download e sobre o qual se pretende trabalhar.
\item [\textbf{PASV}] comando que pede ao servidor para ficar à escuta numa porta de dados diferente da porta usada pelo serviço de controlo. Este comando recebe uma resposta que contém o endereço e a porta na qual o servidor ficou à escuta para poder estabelecer uma outra ligação TCP usada para transferência de dados.
\item [\textbf{RETR filename}] comando \texttt{retrieve} que pede ao servidor que inicie a transmissão de uma cópia do ficheiro especificado pelo campo \texttt{filename} usando a nova ligação estabelecida ao endereço e porta recebidos pelo comando anterior.
\end{labeling}

Uma vez estabelecida a ligação dedicada de transmissão de dados, estes vão sendo recebidos de forma ordenada pelo cliente de FTP e armazenados em disco.

\subsection*{Casos de uso}
Um possível caso de uso pode ser o download de um ficheiro de forma anónima do URL \texttt{ftp://ftp.up.pt/pub/CentOS/filelist.gz}:

\begin{figure}[H]
    \center
    \includegraphics[scale=0.45]{res/anonymous.png}
    \caption{Utilização anónima para download de um ficheiro}
    \label{fig:anonymous.png}
\end{figure}

Também foi testado o download de um ficheiro quando é requerida a autenticação do utilizador no servidor de FTP:

\begin{figure}[H]
    \center
    \includegraphics[scale=0.45]{res/authenticated.png}
    \caption{Utilização autenticada no download de um ficheiro}
    \label{fig:authenticated.png}
\end{figure}

\section{Conclusões}
\iffalse (síntese da informação apresentada nas secções anteriores; reflexão sobre os objectivos de aprendizagem alcançados) \fi

O projeto pode ser sumariamente descrito pelo seu principal propósito que é o desenvolvimento de um protocolo de ligação de dados e o seu teste aquando da obtenção de sucesso na transferência de ficheiros entre dois computadores, com mecanismos de recuperação face a algumas situações de erros. 


Findo o projeto, consideramos que atingimos os objetivos definidos tendo também implementado alguns dos elementos de valorização especificados no guião. Esta abordagem prática permitiu também uma melhor consciência do funcionamento e dos problemas inerentes às redes comunicações entre computadores, abordados nas aulas teóricas.

\begin{thebibliography}{9}
\bibitem{lamport93}
  Andrew S. Tanenbaum,
  David J. Wetherall,
  \emph{Computer Networks},
  Prentice Hall, 
  5th edition,
  2011.
\end{thebibliography}

\appendix
\section{Código fonte}
\subsection{Camada de aplicação}
\subsubsection*{netlink.c}
\begin{lstlisting}[style=customcwithlines]
void receiver_stats()
#include <sys/types.h>
#include <sys/stat.h>
#include <fcntl.h>
#include <stdio.h>
#include <string.h>
#include <stdlib.h>
#include <termios.h>
#include <unistd.h>

#include "packets.h"
#include "file.h"
#include "netlink.h"
#include "serial_port.h"

struct file file_to_send;
int max_retries = 3;

void help(char **argv)
{
	fprintf(stderr, "Usage: %s [OPTIONS] <serial port>\n", argv[0]);
	fprintf(stderr, "\n Program options:\n");
	fprintf(stderr, "  -t <FILEPATH>\t\ttransmit file over the serial port\n");
	fprintf(stderr, "  -i\t\t\ttransmit data read from stdin\n");
	fprintf(stderr, "  -b <BAUDRATE>\t\tbaudrate of the serial port\n");
	fprintf(stderr,
			"  -p <DATASIZE>\t\tmaximum bytes of data transfered each frame\n");
	fprintf(stderr, "  -r <RETRY>\t\tnumber of retry attempts\n");
}

int parse_serial_port_arg(int index, char **argv)
{
	if ((strcmp("/dev/ttyS0", argv[index]) != 0)
			&& (strcmp("/dev/ttyS1", argv[index]) != 0)
			&& (strcmp("/dev/ttyS4", argv[index]) != 0)) {
		fprintf(stderr, "Error: bad serial port value\n");
		return -1;
	}

	return index;
}

int parse_baudrate_arg(int baurdate_index, char **argv)
{
	if (strcmp("B50", argv[baurdate_index]) == 0) {
		serial_port_baudrate = B50;
		return 0;
	} else if (strcmp("B75", argv[baurdate_index]) == 0) {
		serial_port_baudrate = B75;
		return 0;
	} else if (strcmp("B110", argv[baurdate_index]) == 0) {
		serial_port_baudrate = B110;
		return 0;
	} else if (strcmp("B134", argv[baurdate_index]) == 0) {
		serial_port_baudrate = B134;
		return 0;
	} else if (strcmp("B150", argv[baurdate_index]) == 0) {
		serial_port_baudrate = B150;
		return 0;
	} else if (strcmp("B200", argv[baurdate_index]) == 0) {
		serial_port_baudrate = B200;
		return 0;
	} else if (strcmp("B300", argv[baurdate_index]) == 0) {
		serial_port_baudrate = B300;
		return 0;
	} else if (strcmp("B600", argv[baurdate_index]) == 0) {
		serial_port_baudrate = B600;
		return 0;
	} else if (strcmp("B1200", argv[baurdate_index]) == 0) {
		serial_port_baudrate = B1200;
		return 0;
	} else if (strcmp("B1800", argv[baurdate_index]) == 0) {
		serial_port_baudrate = B1800;
		return 0;
	} else if (strcmp("B2400", argv[baurdate_index]) == 0) {
		serial_port_baudrate = B2400;
		return 0;
	} else if (strcmp("B4800", argv[baurdate_index]) == 0) {
		serial_port_baudrate = B4800;
		return 0;
	} else if (strcmp("B9600", argv[baurdate_index]) == 0) {
		serial_port_baudrate = B9600;
		return 0;
	} else if (strcmp("B19200", argv[baurdate_index]) == 0) {
		serial_port_baudrate = B19200;
		return 0;
	} else if (strcmp("B38400", argv[baurdate_index]) == 0) {
		serial_port_baudrate = B38400;
		return 0;
	}
	fprintf(stderr, "Error: bad serial port baudrate value\n");
	fprintf(stderr,
			"Valid baudrates: B110, B134, B150, B200, B300, B600, B1200, B1800, B2400, B4800, B9600, B19200, B38400\n");
	return -1;
}

void parse_max_packet_size(int packet_size_index, char **argv)
{
	int val = atoi(argv[packet_size_index]);
	if (val > FRAME_SIZE || val < 0)
		max_data_transfer = FRAME_SIZE;
	else
		max_data_transfer = val;

#ifdef NETLINK_DEBUG_MODE
	fprintf(stderr,"\nparse_max_packet_size:\n");
	fprintf(stderr,"  max_packet_size=%d\n", max_data_transfer);
#endif
}

void parse_max_retries(int packet_size_index, char **argv)
{
	int val = atoi(argv[packet_size_index]);
	if (val <= 0)
		max_retries = 4;
	else
		max_retries = 1 + val;

#ifdef NETLINK_DEBUG_MODE
	fprintf(stderr,"\nmax_retries:\n");
	fprintf(stderr,"  max_retries=%d\n", max_retries);
#endif
}

int parse_flags(int* t_index, int* i_index, int* b_index, int* p_index,
		int* r_index, int argc, char **argv)
{
	for (size_t i = 0; i < (argc - 1); i++) {
		if ((strcmp("-t", argv[i]) == 0)) {
			*t_index = i;
		} else if ((strcmp("-i", argv[i]) == 0)) {
			*i_index = i;
		} else if ((strcmp("-b", argv[i]) == 0)) {
			*b_index = i;
		} else if ((strcmp("-p", argv[i]) == 0)) {
			*p_index = i;
		} else if ((strcmp("-r", argv[i]) == 0)) {
			*r_index = i;
		} else if ((argv[i][0] == '-')) {
			return -1;
		}
	}
#ifdef NETLINK_DEBUG_MODE
	fprintf(stderr,"\nparse_flags(): flag indexes\n");
	fprintf(stderr,"  -t=%d\n  -i=%d\n  -b=%d\n  -p=%d\n  -r=%d\n", *t_index, *i_index, *b_index,
			*p_index, *r_index);
#endif
	return 0;
}

int parse_args(int argc, char **argv, int *is_transmitter)
{

#ifdef NETLINK_DEBUG_MODE
	fprintf(stderr,"\nparse_args(): received arguments\n");
	fprintf(stderr,"  argc=%d\n  argv=%s\n", argc, *argv);
#endif

	if (argc < 2) {
		return -1;
	}

	if (argc == 2)
		return parse_serial_port_arg(1, argv);

	int t_index = -1, i_index = -1, b_index = -1, p_index = -1, r_index = -1;

	if (parse_flags(&t_index, &i_index, &b_index, &p_index, &r_index, argc,
			argv)) {
		fprintf(stderr, "Error: bad flag parameter\n");
		return -1;
	}

	if (t_index > 0 && t_index < argc - 1) {
		if (read_file_from_disk(argv[t_index + 1], &file_to_send) < 0) {
			return -1;
		}
		*is_transmitter = 1;
	} else {
		if (i_index > 0 && i_index < argc - 1) {
			if (read_file_from_stdin(&file_to_send) < 0) {
				return -1;
			}
			*is_transmitter = 1;
		}
	}

	if (b_index > 0 && b_index < argc - 1) {
		if (parse_baudrate_arg(b_index + 1, argv) != 0) {
			return -1;
		}
	}

	if (p_index > 0 && p_index < argc - 1) {
		parse_max_packet_size(p_index + 1, argv);
	}

	if (r_index > 0 && r_index < argc - 1) {
		parse_max_retries(r_index + 1, argv);
	}

	return parse_serial_port_arg(argc - 1, argv);
}

int main(int argc, char **argv)
{
	int port_index = -1;
	int is_transmitter = 0;

	if ((port_index = parse_args(argc, argv, &is_transmitter)) < 0) {
		help(argv);
		exit(EXIT_FAILURE);
	}

	if (is_transmitter) {
		fprintf(stderr, "transmitting %s\n", file_to_send.name);
		return send_file(argv[port_index], &file_to_send, max_retries);
	} else {
		fprintf(stderr, "receiving file\n");
#ifdef NETLINK_DEBUG_MODE
		fprintf(stderr, "\tserial_port_baudrate:%d\n", serial_port_baudrate);
		fprintf(stderr, "\tis_transmitter:%d\n", is_transmitter);
#endif
		return receive_file(argv[port_index], max_retries);
	}
}
\end{lstlisting}


\end{document}

